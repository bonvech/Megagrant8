%\section{Публикации по теме заявки}
% https://ui.adsabs.harvard.edu/abs/2009ApJ...697.1071A/abstract
%https://www.doi2bib.org/

Aab A., et al. (The Pierre Auger Collaboration), 2014, Phys. Rev. D, 90, 122006

--- 2015, NIM A 798, 172

Aartsen M.G., et al. (IceCube Collaboration), 2013, Measurement of the cosmic ray energy spectrum with IceTop-73. Phys. Rev. D88, 
042004.

Aartsen M.G., et al. (The IceCube Collaboration*, \fermilat{}, MAGIC, AGILE, ASAS-SN, HAWC, H.E.S.S, INTEGRAL, Kanata, Kiso, Kapteyn, Liverpool Telescope, Subaru, Swift/NuSTAR, VERITAS, and VLA/17B-403 teams), 2019, Multimessenger observations of a flaring blazar coincident with high-energy neutrino IceCube-170922A. Science 361, n. 6398, eaat1378

Abbasi R.U., et al. (High Resolution Fly's Eye Collaboration), 2005, A Study of the Composition of Ultra-High-Energy Cosmic Rays Using the High-Resolution Fly's Eye. APJ. 622. 910

--- 2008, First Observation of the Greisen-Zatsepin-Kuzmin Suppression. Phys. Rev. Lett., 100, 101101

--- 2013, APh 44, 40

--- 2015, Astropart. Phys., 64, 49

--- 2018, The Cosmic Ray Energy Spectrum between 2 PeV and 2 EeV Observed with the TALE Detector in Monocular Mode, ApJ, 865, 74

Abbott, B. P., Abbott, R., Abbott, T. D., et al. 2016a, Localization and broadband follow-up of the gravitational-wave transient GW150914, ApJL 826, L13

--- 2016b, Observation of Gravitational Waves from a Binary Black Hole Merger, Physical Review Letters, 116, 061102

Abbott, B. P. et al., 2017a, Multi-messenger Observations of a Binary Neutron Star Merger, ApJL 848, L12

Abbott, B. P. et al., 2017b, Gravitational Waves and $\gamma$-Rays from a Binary Neutron Star Merger: GW170817 and GRB 170817A, ApJL 848, L13

Abdo, A. A., Ackermann, M., Ajello, M., et al. 2009a, \fermi{} Large Area Telescope Measurements of the Diffuse $\gamma$-Ray Emission at Intermediate Galactic Latitudes, Phys. Rev. Lett., 103, 251101

--- 2009b, \fermilat{} Observation of Diffuse Gamma Rays Produced Through Interactions Between Local Interstellar Matter and High-energy Cosmic Rays, ApJ, 703, 1249

--- 2009c, The on-orbit calibration of the \fermi{} Large Area Telescope, Astroparticle Physics, 32, 193

--- 2010a, \fermi{} Large Area Telescope First Source Catalog, ApJS, 188, 405

--- 2010b, \fermi{} Observations of Cassiopeia and Cepheus: Diffuse $\gamma$-ray Emission in the Outer Galaxy, ApJ, 710, 133

--- 2010c, Spectrum of the Isotropic Diffuse $\gamma$-Ray Emission Derived from First-Year \fermi{} Large Area Telescope Data, Phys. Rev. Lett., 104, 101101

--- 2011, $\gamma$-Ray Flares from the Crab Nebula, Science, 331, 739

--- 2013, The Second \fermi{} Large Area Telescope Catalog of $\gamma$-Ray Pulsars, ApJS, 208, 17

Abdollahi, S., et al., 2017, Cosmic-ray electron-positron spectrum from 7 GeV to 2 TeV with the \fermi{} Large Area Telescope, Phys. Rev. D, 95, 082007

Abe, K., Fuke, H., Haino, S., et al. 2012, Measurement of the Cosmic-Ray Antiproton Spectrum at Solar Minimum with a Long-Duration Balloon Flight over Antarctica, Physical Review Letters, 108, 051102

Abeysekara, A. U., et al. 2013, Sensitivity of the high-altitude water Cherenkov detector to sources of multi-TeV gamma rays, Astroparticle Physics, 50, 26

--- 2017a, The 2HWC HAWC Observatory $\gamma$-Ray Catalog, ApJ, 843, 40

--- 2017b, Extended $\gamma$-ray sources around pulsars constrain the origin of the positron flux at Earth, Science, 358, 911

Abdo A. A., et al., 2009, Measurement of the Cosmic Ray $e^+ + e^-$ Spectrum from 20 GeV to 1 TeV with the \fermi{} Large Area Telescope, Phys. Rev. Lett., 102, 181101

Abreu P. et al. (Pierre Auger Collaboration). 2011, Search for first harmonic modulation in the right ascension distribution of cosmic rays detected at the Pierre Auger Observatory. Astropart. Phys., 34, 627

--- 2012a, Large-scale Distribution of Arrival Directions of Cosmic Rays Detected Above $10^{18}$ eV at the Pierre Auger Observatory. AJP Sup. Ser. 203. 34

--- 2012b, Antennas for the detection of radio emission pulses from cosmic-ray induced air showers at the Pierre Auger Observatory, JINST 7, P10011.

Abu-Zayyad T. et al. (The Telescope Array Collaboration), 2012, NIM A 689, 87

--- 2013, The Cosmic Ray Energy Spectrum Observed with the Surface Detector of the Telescope Array Experiment. ApJ. 768. L1

Accardo, L., Aguilar, M., Aisa, D., et al. 2014, High Statistics Measurement of the Positron Fraction in Primary Cosmic Rays of 0.5-500 GeV with the Alpha Magnetic Spectrometer on the International Space Station, Physical Review Letters, 113, 121101 

Acero, F., Ackermann, M., Ajello, M., et al. 2016, The First \fermilat{} Supernova Remnant Catalog, ApJS, 224, 8

--- 2015, \fermi{} Large Area Telescope Third Source Catalog, ApJS, 218, 23

Acharya, B. S., Actis, M., Aghajani, T., et al. 2013, Introducing the CTA concept, Astroparticle Physics, 43, 3

Ackermann, M., et al., 2010, \fermilat{} observations of cosmic-ray electrons from 7 GeV to 1 TeV, Phys. Rev. D, 82, 092004

--- 2012a, A Statistical Approach to Recognizing Source Classes for Unassociated Sources in the First \fermilat{} Catalog, ApJ, 753, 83

--- 2012b, \fermilat{} Observations of the Diffuse-Ray Emission: Implications for Cosmic Rays and the Interstellar Medium, ApJ, 750, 3

--- 2012c, Measurement of Separate Cosmic-Ray Electron and Positron Spectra with the \fermi{} Large Area Telescope, Physical Review Letters, 108, 011103

--- 2012d, The \fermi{} Large Area Telescope on Orbit: Event Classification, Instrument Response Functions, and Calibration, ApJS, 203, 4

--- 2013a, Detection of the Characteristic Pion-Decay Signature in Supernova Remnants, Science, 339, 807

--- 2013b, Determination of the Point-spread Function for the \fermi{} Large Area Telescope from On-orbit Data and Limits on Pair Halos of Active Galactic Nuclei, ApJ, 765, 54

--- 2013c, The First \fermilat{} Catalog of Sources above 10 GeV, ApJS, 209, 34

--- 2013d, The First \fermilat{} $\gamma$-Ray Burst Catalog, ApJS, 209, 11

--- 2014a, Inferred Cosmic-Ray Spectrum from \fermi{} Large Area Telescope -Ray Observations of Earth's Limb, Physical Review Letters, 112, 151103

--- 2014b, The Spectrum and Morphology of the \fermi{} Bubbles, ApJ, 793, 64

--- 2015a, The Spectrum of Isotropic Diffuse $\gamma$-Ray Emission between 100 MeV and 820 GeV, ApJ, 799, 86

--- 2015b, The Third Catalog of Active Galactic Nuclei Detected by the \fermi{} Large Area Telescope, ApJ, 810, 14

--- 2016a, 2FHL: The Second Catalog of Hard \fermilat{} Sources, ApJS, 222, 5

--- 2016b, \fermilat{} Observations of the LIGO Event GW150914, ApJ, 823, L2 Addison, P. S. 2002, The Illustrated Wavelet Transform Handbook. IOP Publishing Ltd.

Adriani, O., Barbarino, G. C., Bazilevskaya, G. A., et al. 2009a, An anomalous positron abundance in cosmic rays with energies 1.5-100GeV, Nature, 458, 607

--- 2009b, New Measurement of the Antiproton-to-Proton Flux Ratio up to 100 GeV in the Cosmic Radiation, Phys. Rev. Lett., 102, 051101

--- 2011, PAMELA Measurements of Cosmic-Ray Proton and Helium Spectra, Science, 332, 69

--- 2013, Measurement of the flux of primary cosmic ray antiprotons with energies of 60 MeV to 350 GeV in the PAMELA experiment, Soviet Journal of Experimental and Theoretical Physics Letters, 96, 621

--- 2014, Measurement of Boron and Carbon Fluxes in Cosmic Rays with the PAMELA Experiment, ApJ, 791, 93

--- 2015, The CALorimetric Electron Telescope (CALET) for high-energy astroparticle physics on the International Space Station, Journal of Physics Conference Series, 632, 012023

--- 2017, Ten Years of PAMELA in Space, Riv. Nuovo Cimento, 40, 473

--- 2018, Extended Measurement of the Cosmic-Ray Electron and Positron Spectrum from 11 GeV to 4.8 TeV with the Calorimetric Electron Telescope on the International Space Station, Phys. Rev. Lett., 120, 261102

--- 2019, Direct Measurement of the Cosmic-Ray Proton Spectrum from 50 GeV to 10 TeV with the Calorimetric Electron Telescope on the International Space Station, Physical Review Letters, 122, 181102

Agafonova N. et al., 2009, JINST

--- 2015, Phys.Rev.Lett., 115, 121802

Aglietta M. et al. (The EAS-TOP Collaboration), 1989, NIM A 277 590, 23

--- 2009, Evolution of the Cosmic-Ray Anisotropy Above $10^{14}$ eV. APJ Lett. 692. L130 

Aguilar, M. et al. 2013, First Result from the Alpha Magnetic Spectrometer on the International Space Station: Precision Measurement of the Positron Fraction in Primary Cosmic Rays of 0.5-350 GeV, Physical Review Letters, 110, 141102

--- 2014a. Precision Measurement of the ($e^+ + e^-$) Flux in Primary Cosmic Rays from 0.5 GeV to 1 TeV with the Alpha Magnetic Spectrometer on the International Space Station. Phys. Rev. Lett., 113, 221102

--- 2014b. Electron and Positron Fluxes in Primary Cosmic Rays Measured with the Alpha Magnetic Spectrometer on the International Space Station. Phys. Rev. Lett., 113, 121102

--- 2015a, Precision Measurement of the Helium Flux in Primary Cosmic Rays of Rigidities 1.9 GV to 3 TV with the Alpha Magnetic Spectrometer on the International Space Station, Phys. Rev. Lett., 115, 211101

--- 2015b, Precision Measurement of the Proton Flux in Primary Cosmic Rays from Rigidity 1 GV to 1.8 TV with the Alpha Magnetic Spectrometer on the International Space Station, Physical Review Letters, 114, 171103

--- 2016a, Antiproton Flux, Antiproton-to-Proton Flux Ratio, and Properties of Elementary Particle Fluxes in Primary Cosmic Rays Measured with the Alpha Magnetic Spectrometer on the International Space Station. Phys. Rev. Lett., 117, 091103

--- 2016b, Precision Measurement of the Boron to Carbon Flux Ratio in Cosmic Rays from 1.9 GV to 2.6 TV with the Alpha Magnetic Spectrometer on the International Space Station. Phys. Rev. Lett., 117, 231102

--- 2017, Observation of the Identical Rigidity Dependence of He, C, and O Cosmic Rays at High Rigidities by the Alpha Magnetic Spectrometer on the International Space Station, Phys. Rev. Lett., 119, 251101

--- 2018a, Observation of New Properties of Secondary Cosmic Rays Lithium, Beryllium, and Boron by the Alpha Magnetic Spectrometer on the International Space Station, Phys. Rev. Lett., 120, 021101

--- 2018b, Precision Measurement of Cosmic-Ray Nitrogen and its Primary and Secondary Components with the Alpha Magnetic Spectrometer on the International Space Station, Phys. Rev. Lett., 121, 051103

--- 2019a, Towards Understanding the Origin of Cosmic-Ray Positrons. Phys. Rev. Lett., 122, 041102

--- 2019b, Towards Understanding the Origin of Cosmic-Ray Electrons. Phys. Rev. Lett., 122, 101101

Aharonian, F., et al., 2008, Energy Spectrum of Cosmic- Ray Electrons at TeV Energies, Physical Review Letters, 101, 261104

--- 2009, Probing the ATIC peak in the cosmic-ray electron spectrum with H.E.S.S., Astron. Astrophys., 508, 561 

Ahn, H. S., Allison, P., Bagliesi, M. G., et al. 2010a, Discrepant Hardening Observed in Cosmic-ray Elemental Spectra, ApJ, 714, L89

--- 2010b, Measurements of the Relative Abundances of High-energy Cosmic-ray Nuclei in the TeV/Nucleon Region, ApJ, 715, 1400

Ajello, M., Albert, A., Atwood, W. B., et al. 2016a, \fermilat{} Observations of High-Energy $\gamma$-Ray Emission toward the Galactic Center, ApJ, 819, 44

--- 2016b. Phys. Rev. Lett., 116, 151105

Aleksic, J., Ansoldi, S., Antonelli, L. A., et al. 2016, The major upgrade of the MAGIC telescopes, Part II: A performance study using observations of the Crab Nebula, Astroparticle Physics, 72, 76

Alfaro, R. et al., 2017, All-particle cosmic ray energy spectrum measured by the HAWC experiment from 10 to 500 TeV, Phys. Rev. D, 96, 122001

Ambrosi, G., et al., 2018, Direct detection of a break in the teraelectronvolt cosmic-ray spectrum of electrons and positrons, Nature, 552, 63

Amenomori, M. et al. 2008, The All-Particle Spectrum of Primary Cosmic Rays in the Wide Energy Range from $10^{14}$ to $10^{17}$ eV Observed with the Tibet-III Air-Shower Array. APJ. 678. 1165

Antoni T. et al. 2004, Large-Scale Cosmic-Ray Anisotropy KASCADE. APJ. 604, 687

--- 2005, KASCADE measurements of energy spectra for elemental groups of cosmic rays: Results and open problems. Astropart. Phys. 24. 1

Antonov R.A. et al. 2006, Antarctic balloon-borne detector of high-energy cosmic rays (SPHERE project), Radiation Physics and Chemistry. 75. 887--890

--- 2013a, Results on the primary CR spectrum and composition reconstructed with the SPHERE-2 detector, Journal of Physics: Conference Series, 409, 12094

--- 2013b, First detailed reconstruction of the primary cosmic ray energy spectrum using reflected Cherenkov light, in Proc. of 33rd International Cosmic Ray Conference, Rio de Janeiro

--- 2015a, Results and prospects on registration of reflected Cherenkov light of EAS from cosmic particles above $10^{15}$ eV, DESY-PROC-2014-04. 411

--- 2015b, Detection of reflected Cherenkov light from extensive air showers in the SPHERE experiment as a method of studying superhigh energy cosmic rays, Phys. Part. Nucl. 46, 60

--- 2015c, Event-by-event study of CR composition with the SPHERE experiment using the 2013 data, arXiv: 1503.04998

--- 2019, Spatial and temporal structure of EAS reflected Cherenkov light signal, Astroparticle Physics, 108, 24

Aoki, M. 1965, On some convergence question in Bayesian optimization problem. IEEE Trans. on Automat. Control. 2. 180-182

Apanasenko A.V. et al. (RUNJOB Collaboration). 2001, Astroparticle Physics, 16, 13 

Apel W.D. et al. (KASCADE-Grande Collaboration). 2010, NIM 595 A 620, 202

--- 2011, Kneelike Structure In The Spectrum Of The Heavy Component Of Cosmic Rays Observed With KASCADE--GRANDE. Phys. Rev. Lett., 107, 171104

--- 2012, The spectrum of high-energy cosmic rays measured with KASCADE-Grande, Astropat. Phys., 36, 183

--- 2013, Ankle-like feature in the energy spectrum of light elements of cosmic rays observed with KASCADE-Grande. Phys. Rev. D. 87. 081101(R).

Asakimori K. et al. 1991, Proc. 22th ICRC, 2, 57

--- 1995, Proc. 24th ICRC, 2, 707

--- 1998, Cosmic-Ray Proton and Helium Spectra: Results from the JACEE Experiment, ApJ, 502, 278

Atkin, E., et al., 2018, New Universal Cosmic-Ray Knee near a Magnetic Rigidity of 10 TV with the NUCLEON Space Observatory, JETP Letters, 108, 5

Atwood, W. B., Abdo, A. A., Ackermann, M., et al. 2009, The Large Area Telescope on the \fermi{} $\gamma$-Ray Space Telescope Mission, ApJ, 697, 1071

--- 2013, Pass 8: Toward the Full Realization of the \fermilat{} Scientific Potential, ArXiv e-prints

Baklagin S.A. et al., 2016, International Journal of Innovative Research in Science, Engineering and Technology, 5, 12229

Barwick, S. W., Beatty, J. J., Bhattacharyya, A., et al. 1997, Measurements of the Cosmic-Ray Positron Fraction from 1 to 50 GeV, ApJ, 482, L191

Batraev, V.V. \& Galkin, V.I. 2018, The separation of air showers by the masses of the primary particles on the basis of the measured angular distributions of Cherenkov light at mountain level, Memoirs of the Faculty of Physics, N 3.

Beatty, J. J., Bhattacharyya, A., Bower, C., et al. 2004, New Measurement of the Cosmic-Ray Positron Fraction from 5 to 15GeV, Physical Review Letters, 93, 241102

Beck, R. 2001, Galactic and Extragalactic Magnetic Fields, Space Sci. Rev., 99, 243 Bell, A.R., 1978. MNRAS, 182, 147

--- 2004. MNRAS, 353, 550

--- 2015. MNRAS, 447, 222

Benitez, R., Bolуs, V.J., Ramirez, M.E. 2010, A wavelet-based tool for studying non- periodicity. Computers and Mathematics with Applications, 60. 634-641

Bennett, C. L., Bay, M., Halpern, M., et al. 2003, The Microwave Anisotropy Probe Mission, ApJ, 583, 1

Berezhko E.G. 1986, Sov. Astr. Lett. 12, 352. (in Russian)

--- 1994, Astroparticle Physics 2, 215

--- 2008, Proc. 30th ICRC, Mexico 2, 109

Berezhnev S. et al. (Tunka Collaboration), 2012, NIM A, 692, 98

Berezinsky V. et al. 2005, Dip in UHECR spectrum as signature of proton interaction with CMB. Phys. Lett. B612. 147

--- 2010, Astropart.Phys., 84, 52

--- 2016, Phys.Rev. D, 94, 023007

Berezinsky V., Kalashev O. 2016, Cascade photons as test of protons in UHECR, Astroparticle Physics, 84, 52

Berger, J.O., Haff, L.R. 1983, A class of minimax estimators of a normal mean vector for arbitrary quadratic loss and unknown covariance matrix. Statist. Decisions. 1. 105-129

Bergman, D.R. \& Belz, J.W. 2007, Cosmic rays: the Second Knee and beyond. Journal of Physics G: Nuclear and Particle Physics. 34. R359

Berne, O., Joblin, C., Tielens, A., Deville, Y., Puigt, M., Guidara, R., Hosseini, S., Mulas, G. Cami, J. 2009, Source separation algorithms for the analysis of hyper-spectral observations of very small interstellar dust particles. First Workshop on Hyperspectral Image and Signal Processing: Evolution in Remote Sensing

Bertsch, D. L., Dame, T. M., Fichtel, C. E., et al. 1993, Diffuse $\gamma$-Ray Emission in the Galactic Plane from Cosmic-Ray, Matter, and Photon Interactions, ApJ, 416, 587

Beuermann, K., Kanbach, G., \& Berkhuijsen, E. M. 1985, Radio structure of the Galaxy -- Thick disk and thin disk at 408 MHz, Astron. Astrophys., 153, 17

Bianchi, S. 2008, Dust extinction and emission in a clumpy galactic disk. An application of the radiative transfer code TRADING, Astron. Astrophys., 490, 461

Binns, W. R., Wiedenbeck, M. E., Arnould, M., et al. 2005, Cosmic-Ray Neon, Wolf-Rayet Stars, and the Superbubble Origin of Galactic Cosmic Rays, ApJ, 634, 351

--- 2014, The Super-TIGER Instrument: Measurement of Elemental Abundances of Ultra- Heavy Galactic Cosmic Rays, ApJ, 788, 18

--- 2016, Observation of the 60Fe nucleosynthesis-clock isotope in galactic cosmic rays, Science, 352, 677

Blandford, R. D., Ostriker, J. P., 1978, Particle acceleration by astrophysical shocks, ApJL 221, L29 

Blandford R.D. 1993, Phys.Scr. 85, 191

--- 2014, Nuclear Physics B (Proc. Suppl.), 256, 9

Blasi, P. 2008, Rapporteur talk: Direct Measurements, Acceleration and Propagation of Cosmic Rays, in Proc. 30th Int. Cosmic Ray Conf. (Merida), arXiv:0801.4534

--- 2009a, Origin of the Positron Excess in Cosmic Rays, Physical Review Letters, 103, 051104

--- 2009b, High-Energy Antiprotons from Old Supernova Remnants, Physical Review Letters, 103, 081103

Blumenthal G. R., 1970, Phys. Rev. D, 1, 1596

Bobik, P., Boella, G., Boschini, M. J., et al. 2013, Cosmic Ray Modulation studied with HelMod Monte Carlo tool and comparison with Ulysses Fast Scan Data during consecutive Solar Minima, ArXiv: 1307.5199

Boezio, M., Carlson, P., Francke, T., et al. 2000, The Cosmic-Ray Electron and Positron Spectra Measured at 1 AU during Solar Minimum Activity, ApJ, 532, 653

--- 2001, The Cosmic-Ray Antiproton Flux between 3 and 49 GeV, ApJ, 561, 787

--- 2012, Chemical composition of galactic cosmic rays with space experiments, Astroparticle Physics, 39, 95

Bolatto, A. D., Wolfire, M., \& Leroy, A. K. 2013, The CO-to-\htwo{} Conversion Factor, ARA\&A, 51, 207

Borisov A.S. \& Galkin V.I. 2013, Design of a Cherenkov telescope for the measurement of PCR composition above 1 PeV, J. Phys.: Conf.Ser. 409, N1, p.012089.

Borog, V.V., Kryanev, A.V., Udumian, D.K. 2009, Combined method for detection of hidden anomalies in chaotic time series (in Russian). Proceeding of International scientific conference ``Nonlinear processes and systems modeling''. Volume 2. Moscow

Boschini, M. J., Della Torre, S., Gervasi, M., et al. 2017, Solution of Heliospheric Propagation: Unveiling the Local Interstellar Spectra of Cosmic-ray Species, ApJ, 840, 115

--- 2018a, HelMod in the Works: From Direct Observations to the Local Interstellar Spectrum of Cosmic-Ray Electrons, ApJ, 854, 94

--- 2018b, Deciphering the Local Interstellar Spectra of Primary Cosmic-Ray Species with HELMOD, ApJ, 858, 61

Bretthorst, G. Larry. 1988, Bayesian Spectrum Analysis and Parameter Estimation. Springer- Verlag Berlin Heidelberg

Broderick, Avery E. et al. 2012, The Cosmological Impact of Luminous TeV Blazars. I. Implications of Plasma Instabilities for the Intergalactic Magnetic Field and Extragalactic $\gamma$- Ray Background. ApJ. 752. id22

Brodskii, B.E., Darkhovskii, B.S. 1988, A Nonparametric Method for Fastest Detection of a Change in the Mean of a Random Sequence. Theory Probab. Appl. 32(4). 640-648

--- 1990, Comparative analysis of some nonparametric methods for fast detection of the change point of a random sequence (in Russian). Theory of Probability and its Applications. 35:4. 639--652

--- 1993, A posteriori detection of multiple change points in a random sequence (in Russian), Automation and Remote Control. 54:1. 54-59

--- 1995a, Asymptotically optimal methods in a problem of the fastest detection of a change point. II. Studies of fastest detection methods (in Russian). Automation and Remote Control. 9. 60--72

--- 1995b, Asymptotically optimal methods in a problem for the fastest detection of a change point. II. Characteristics of methods for the fastest detection of a change point (in Russian), Automation and Remote Control. 10. 50--59

--- 1998, Nonparametric segmentation of the electrical signals of the brain (in Russian), Automation and Remote Control. 59:2. 172-179

Budnev N., et al., (Tunka Collaboration), 2013, Astropart. Phys., 50, 18

--- 2018, PoSICRC2017,768

Buehler, R., Scargle, J. D., Blandford, R. D., et al. 2012, $\gamma$-Ray Activity in the Crab Nebula: The Exceptional Flare of 2011 April, ApJ, 749, 26

Bulatov, V., Filipov, S., Karmanov, D., et al., 2018. NUCLEON-2 mission for the investigation of heavy cosmic rays' nuclei. Journal of Instrumentation, 13, P11021

Burkatovskaya, Yu. B., Markov, N. G., Morozov, A. S., Serykh, A. P. 2007а, Application of Johnson distribution to the problem of aerospace images classification. Bulletin of the Tomsk Polytechnic University 311(5). 69--73

--- 2007b, Application of Johnson distributions in image processing (in Russian). Fourteen Russian School-colloquium on Statistics Methods. Moscow. Review of applied and industrial mathematics. 1094--1095

--- 2015, CUSUM Algorithms for Parameter Estimation in Queueing Systems with Jump Intensity of the Arrival Process. Communications in Computer and Information Science (CCIS). Queueing Theory and Applications. 564. 275-288.

Burkatovskaya, Yu. B., Kabanova, T., Tokareva, O., 2016, Sign CUSUM Algorithm for Change-Point Detection of the MMPP Controlling Chain State, Information Technologies and Mathematical Modelling -- Queueing Theory and Applications, Springer, 18

Bykov, A.M., Ellison, D.C., Osipov, S.M., Vladimirov, A.E., 2014, ApJ 789, 137

Camps, P., \& Baes, M. 2015, SKIRT: An advanced dust radiative transfer code with a user- friendly architecture, Astronomy and Computing, 9, 20

Caprioli, D., \& Spitkovsky, A., 2014a, ApJ 783, 91

--- 2014b, ApJ 794, 46

--- 2014c, ApJ 794, 47

Chan, M. H., Leung, C. H., 2018, Constraining dark matter by the 511 keV line, MNRAS, 479, 2229

Chang, J., Ambrosi, G., An, Q., et al. 2017, The DArk Matter Particle Explorer mission, Astroparticle Physics, 95, 6

Chen, W. et al. 2015, Phys. Rev. Lett., 115, 211103

Chernyshov, D.O., Cheng,K.S., Dogiel, V.A., Ko, C.M. 2014, Nuclear Physics B (Proc. Suppl.), 256, 179

Chudakov A.E., 1974, Proc. All-USSR Symp. on Exp. Meth. Of UHECR,Yakutsk, 69. (In Russian)

Chi, X., \& Wolfendale, A. W. 1991, The interstellar radiation field: a datum for cosmic ray physics, Journal of Physics G Nuclear Physics, 17, 987

Cirelli M., Taoso M. 2016, Updated galactic radio constraints on Dark Matter. Journal of Cosmology and Astroparticle Physics. 7. 041

Connaughton, V., Burns, E., Goldstein, A., et al. 2016, \fermi{} GBM Observations of LIGO Gravitational Wave event GW150914, ArXiv e-prints

Cordes J. M., Lazio T. J. W. 2002, NE2001.I. A New Model for the Galactic Distribution of Free Electrons and its Fluctuations. arXiv:astro-ph/0207156

Cummings, A., Stone, E., Heikkila, B. C., et al. 2015, Voyager 1 Observations of Galactic Cosmic Rays in the Local Interstellar Medium: Energy Density and Ionization Rates, Proc. 34th ICRC (Hague), 318

Cummings, A., Stone, E., Heikkila, B. C., et al. 2016, Galactic Cosmic Rays in the Local Interstellar Medium: Voyager 1 Observations and Model Results, ApJ, 831, 18.

Dame, T. M., Hartmann, D., \& Thaddeus, P. 2001, The Milky Way in Molecular Clouds: A New Complete CO Survey, ApJ, 547, 792

Daubechies, I. 1992. Ten Lectures on Wavelets. Society for Industrial and Applied Mathematics

Deharveng, L., Pe\~{n}a, M., Caplan, J., et al. 2000, Oxygen and helium abundances in Galactic Hii regions -- II. Abundance gradients, MNRAS, 311, 329

Dermer, C. D. 1986a, Binary collision rates of relativistic thermal plasmas. II -- Spectra, ApJ, 307, 47

--- 1986b, Secondary production of neutral pi-mesons and the diffuse galactic gamma radiation, Astron. Astrophys., 157, 223

--- 2011, ApJ Lett., 733, L21

Devjatykh, D., Gerget, O., Berestneva, O. 2014, Sleep Apnea Detection Based on Dynamic Neural Networks. Communications in Computer and Information Science. 466. 556?567

Derbina V.A. et.al. (RUNJOB Collaboration). 2005, Cosmic-Ray Spectra and Composition in the Energy Range of 10--1000 TeV per Particle Obtained by the RUNJOB Experiment. APJ Lett. 628. L41

Dzhatdoev, T. A. et al. 2017, Electromagnetic cascade masquerade: a way to mimic $\gamma$- axion-like particle mixing effects in blazar spectra. A\&A. 603. A59

Dobrovidov A.V., Koshkin G.M., Vasiliev V.A. 2012, Nonparametric State Space Models. Heber, USA: Kendrick Press. Nonparametric estimation of functionals of stationary sequences distributions. Moscow.: Nauka, 2004 -- 508 p. www.naukaran.ru (in Russian)

Drury, L.O'C., Falle, S.A.E.G., 1986, MNRAS, 223, 353

Durrer R., Neronov A. 2013, Cosmological magnetic fields: their generation, evolution and observation. The Astronomy and Astrophysics Review. 21. 62

Dwek, E., Arendt, R. G., Fixsen, D. J., et al. 1997, Detection and Characterization of Cold Interstellar Dust and Polycyclic Aromatic Hydrocarbon Emission, from COBE Observations, ApJ, 475, 565

Efron B., Morris C. 1976, Families of minimax estimators of the mean of a multivariate normal distribution. Ann. Statist. 4. 11-21

--- 2010, Large-Scale Inference: Empirical Bayes Methods for Estimation, Testing, and Prediction (Institute of Mathematical Statistics Monographs). Cambridge University Press, 1 edition

Essey W. \& Kusenko A. 2010, Secondary Photons and Neutrinos from Cosmic Rays Produced by Distant Blazars. Phys. Rev. Lett. 104. id141102

Fedorova Y., et al. 2007, Proc. 30th International Cosmic Ray Conference, 4, 463

\fermilat{} collaboration, 2019, \fermi{} Large Area Telescope Fourth Source Catalog, arXiv: 1902.10045

Feroz, F., \& Hobson, M. P. 2008, Multimodal nested sampling: an efficient and robust alternative to Markov Chain Monte Carlo methods for astronomical data analyses, MNRAS, 384, 449

--- 2009, MULTINEST: an efficient and robust Bayesian inference tool for cosmology and particle physics, MNRAS, 398, 1601

--- 2013, Importance Nested Sampling and the MultiNest Algorithm, ArXiv: 1306.2144

Ferriere, K., Gillard, W., \& Jean, P. 2007, Spatial distribution of interstellar gas in the innermost 3 kpc of our galaxy, Astron. Astrophys., 467, 611

Finkbeiner, D. P. 2004, Microwave Interstellar Medium Emission Observed by the Wilkinson Microwave Anisotropy Probe, ApJ, 614, 186

Fisher, A. J., Hagen, F. A., Maehl, R. C., et al. 1976, The isotopic composition of cosmic rays with $Z$ between 5 and 26, ApJ, 205, 938

Fowler J.W., et al., 2001, APh 15, 49

Freudenreich, H. T. 1998, A COBE Model of the Galactic Bar and Disk, ApJ, 492, 495 Fukunaga, K. 1972, Introduction to statistical pattern recognition Academic Press, New York and London

Funk, S. , 2015, Ground- and Space-Based $\gamma$-Ray Astronomy, Annual Review of Nuclear and Particle Science, 65, 245

Gaensler, B. M., \& Johnston, S. 1995, The pulsar/supernova remnant connection, MNRAS, 277, 1243

Galkin, V.I., Borisov, A.S., Bakhromzod R., et al. 2018, A Method for Estimation of the Parametersof the Primary Particle of an Extensive Air Shower by a High-Altitude Detector, Moscow University Physics Bulletin, 73, No. 2, pp. 179--186.

Garcia-Munoz, M., Simpson, J. A., \& Wefel, J. P. 1979, The isotopes of neon in the galactic cosmic rays, ApJ, 232, L95

Garyaka A.P. et al., 2008, J. Phys. G: Nucl. Part. Phys. 35, 115201

G\'еnolini, Y., Maurin, D., Moskalenko, I. V., Unger M., 2018, Current status and desired precision of the isotopic production cross sections relevant to astrophysics of cosmic rays: Li, Be, B, C, and N, Phys. Rev. C, 98, 034611

Giacinti, G. et al. 2012, Cosmic ray anisotropy as signature for the transition from galactic to extragalactic cosmic rays. J. Cosmol. Astropart. Phys. 2012. 031

Ginzburg, V. L., 1999, What problems of physics and astrophysics seem now to be especially important and interesting (thirty years later, already on the verge of XXI century)?, Physics Uspekhi 42, 353

Girshick M. A., Rubin H. A. 1952, Bayes approach to a quality control model. Ann. Math. Statist. 23(1). 114-125.

Gleser L.J. 1986, Minimax estimators of a normal mean vector for arbitrary quadratic loss and unknown covariance matrix. Ann. Statist. 14. 1625-1633

Golden, R. L., Stochaj, S. J., Stephens, S. A., et al. 1996, Measurement of the Positron to Electron Ratio in Cosmic Rays above 5 GeV, ApJ, 457, L103

Golyandina, N., Nekrutkin, V., Zhigljavsky, A. 2001, Analysis of Time Series Structure: SSA and related techniques. Chapman and Hall/CRC

--- 2013, Singular Spectrum Analysis for time series. Springer Briefs in Statistics, Springer.

Gonzalez, R.E., Woods, R.E. 2002, Digital image processing. Prentice Hall, ISBN 0201180758, 9780201180756

Gorski, K. M., Hivon, E., Banday, A. J., et al. 2005, HEALPix: A Framework for High- Resolution Discretization and Fast Analysis of Data Distributed on the Sphere, ApJ, 622, 759

Graff, P., Feroz, F., Hobson, M. P., et al. 2012, BAMBI: blind accelerated multimodal Bayesian inference, MNRAS, 421, 169

--- 2014, SKYNET: an efficient and robust neural network training tool for machine learning in astronomy, MNRAS, 441, 1741

Graves, A., Mohamed, A., Hinton, G. 2013, Speech Recognition with Deep Recurrent Neural Networks. Acoustics, Speech and Signal Processing (ICASSP), 2013 IEEE International Conference on: 6645-6649.

Greizen K., 1966, End to the Cosmic-Ray Spectrum? Phys. Rev. Lett. 16, 748. Grigorov N.L. et al. 1971, Proc. 12th ICRC, 5, 1760

Hahn, G.J., Shapiro, S.S. 1967, Statistical Models in Engineering. New York: John Wiley \& Sons, ISBN 0471339156, 9780471339151

Haverkorn, M., Brown, J. C., Gaensler, B. M., et al. 2008, The Outer Scale of Turbulence in the Magnetoionized Galactic Interstellar Medium, ApJ, 680, 362

Herms, J., Ibarra, A., Vittino, A., Wild, S., 2017, Antideuterons in cosmic rays: sources and discovery potential, JCAP, 02, 018

Higdon, J. C., \& Lingenfelter, R. E. 2003, The Superbubble Origin of 22Ne in Cosmic Rays, ApJ, 590, 822

Hillas A.M. 1984, Ann. Rev. Astron. Astrophys, 22, 425

Hinton, J. A., \& Hofmann, W. 2009, Teraelectronvolt Astronomy, ARA\&A, 47, 523 

Hochreiter, S., Bengio, Y., Frasconi, P., Schmidhuber, J. 2001, Gradient flow in recurrent nets: the difficulty of learning long-term dependencies. In S. C. Kremer and J. F. Kolen, editors, A Field Guide to Dynamical Recurrent Neural Networks. IEEE Press

Holder, J., Atkins, R. W., Badran, H. M., et al. 2006, The first VERITAS telescope, Astroparticle Physics, 25, 391

Horns M. \& Meyer D. 2012, Revisiting the Indication for a low opacity Universe for very high energy $\gamma$-rays. arXiv:1211.6405

--- 2012, First lower limits on the photon-axion-like particle coupling from very high energy $\gamma$-ray observations. Phys. Rev. D.87. id. 035027

Hunter, S. D., Bertsch, D. L., Catelli, J. R., et al. 1997, EGRET Observations of the Diffuse $\gamma$-Ray Emission from the Galactic Plane, ApJ, 481, 205

Hyvarinen, A., Oja, E. 2000. Independent component analysis: Algorithms and applications (PDF). Neural Networks 13 (4-5): 411-430

--- 2010, Independent component analysis of short-time Fourier transforms for spontaneous EEG/MEG analysis. NeuroImage 49. 257-271

Istomin Ya.N., 2011, New Astronomy, Volume 27, p. 13-18 

Ivanenko I.P. et al. 1993, Bul. Rus. Ac. Nauk. Ser. Phys., 57, 76

Ivanov A.A. et al., 2009, New J. Phys. 35, 115201

--- 2013, EPJ Web of Conferences 53, 04003

Jaffe, T. R., Leahy, J. P., Banday, A. J., et al. 2010, Modelling the Galactic magnetic field on the plane in two dimensions, MNRAS, 401, 1013

---  2011, Connecting synchrotron, cosmic rays and magnetic fields in the plane of the Galaxy, MNRAS, 416, 1152

James, W., Stein, C. 1961, Estimation with quadratic loss. Proc. Fourth Berkeley Symp. Math. Statist. Prob. 1, pp. 361--379

Jansson, R., \& Farrar, G. R. 2012, A New Model of the Galactic Magnetic Field, ApJ, 757, 14

J\'ohannesson, G., Moskalenko, I. V., Orlando, E., et al. 2015, The Effects of Three Dimensional Structures on Cosmic-Ray Propagation and Interstellar Emissions, Proc. 34th ICRC (Hague), 517

--- 2016, Bayesian analysis of cosmic-ray propagation: evidence against homogeneous diffusion, ApJ, 824, 16

J\'ohannesson, G., Porter, T. A., Moskalenko, I. V., 2018, The Three-dimensional Spatial Distribution of Interstellar Gas in the Milky Way: Implications for Cosmic Rays and High-energy $\gamma$-ray Emissions, ApJ, 856, 45

--- 2019, Cosmic-Ray Propagation in Light of the Recent Observation of Geminga, ApJ, 879, 2

Johnson, N.L. 1949, Systems of Frequency Curves Generated by Methods of Translation, Biometrika 36 (1/2): 149-176

--- 1965, Tables to Facilitate Fitting SV Frequency Curves. Oxford: Biometrika Trust. 52-57

--- 1983, Encyclopedia of Statistical Sciences. pp. 303-314. New York: John Wiley \& Sons 

--- Jokipii, J. R., Levy, E. H., \& Hubbard, W. B. 1977, Effects of particle drift on cosmic-ray transport. I -- General properties, application to solar modulation, ApJ, 213, 861

Jones, F. C. 1968, Calculated Spectrum of Inverse-Compton-Scattered Photons, Physical Review, 167, 1159

--- 2001a, The Modified Weighted Slab Technique: Models and Results, ApJ, 547, 264

--- 2001b, K-capture cosmic ray secondaries and reacceleration, Adv. Space Res., 27, 737 

--- Jung, T.-P., Makeig, S., Lee, T.-W., McKeown, M. J., Brown, G., Bell, A. J., Sejnowksi, T. J. 2000, Independent component analysis of biomedical signals. Proc. Second Int. Workshop ICA and BSS. 633-644

Kabanova, T.V., Vorobeinikov, S.E., 2002, Detection of a disorder moment of a sequence of independent random values (in Russian). Radiotekhnika i elektronika, RAS. 47(10). 1198-1203

Kachelriess, M., \& Ostapchenko, S. 2012, Deriving the cosmic ray spectrum from $\gamma$-ray observations, Phys. Rev. D, 86, 043004

--- 2014, Nuclear Enhancement of the Photon Yield in Cosmic Ray Interactions, ApJ, 789, 136

--- 2015, New Calculation of Antiproton Production by Cosmic Ray Protons and Nuclei, ApJ, 803, 54

Kadler, M., Krau\ss, F., Mannheim, K., et al. 2016, Coincidence of a high-fluence blazar outburst with a PeV-energy neutrino event, ArXiv e-prints

Kalberla, P. M. W., Burton, W. B., Hartmann, D., et al. 2005, The Leiden/Argentine/Bonn (LAB) Survey of Galactic HI. Final data release of the combined LDS and IAR surveys with improved stray-radiation corrections, Astron. Astrophys., 440, 775

--- 2009, The Hi Distribution of the Milky Way, ARA\&A, 47, 27

--- 2010, GASS: the Parkes Galactic all-sky survey. II. Stray-radiation correction and second data release, Astron. Astrophys., 521, A17

Kang H., Jones T.W., 2006, Astropart. Phys. 25, 246

Kamae, T., Karlsson, N., Mizuno, T., et al. 2006, Parameterization of $e^\pm$, and Neutrino Spectra Produced by p-p Interaction in Astronomical Environments, ApJ, 647, 692

Karwin, C., Murgia, S., Tait, T. M. P., Porter, T. A., Tanedo, P., 2017, Dark matter interpretation of the \fermilat{} observation toward the Galactic Center, Phys. Rev. D, 95, 103005

Karwin, C., Murgia, S., Campbell, S., Moskalenko, I. V., 2019, \fermilat{} Observations of $\gamma$-Ray Emission Towards the Outer Halo of M31, ApJ, in press (arXiv: 1903.10533) 

Kerp, J., Winkel, B., Ben Bekhti, N., et al. 2011, The Effelsberg Bonn H I Survey (EBHIS), Astronomische Nachrichten, 332, 637

Kobayashi, T., Komori, Y., Yoshida, K., Nishimura, J., 2004, The Most Likely Sources of High-Energy Cosmic-Ray Electrons in Supernova Remnants, ApJ, 601, 340

Kolpak, M. A., Jackson, J. M., Bania, T. M., et al. 2002, The Radial Distribution of Cold Atomic Hydrogen in the Galaxy, ApJ, 578, 868

Korsmeier, M., Donato, F., Fornengo, N., 2018, Prospects to verify a possible dark matter hint in cosmic antiprotons with antideuterons and antihelium, Phys. Rev. D, 97, 103011

Kronberg P. P., Dufton Q. W., Li H., Colgate S. A. 2001, Magnetic Energy of the Intergalactic Medium from Galactic Black Holes, The Astrophysical Journal, 560, 178

Krymskiy G.F., 1977, Soviet Physics-Doklady, 22, 327 (in Russian)

--- 1997, Soviet Physics-Doklady, 234, 1306 (in Russian)

Kulikov, G.V., Khristiansen, G.B. 1959, On the Size Spectrum of Extensive Air Showers, Sov. Phys. JEPT, 35, 441

Kuzmichev L. et al. (TAIGACollaboration), 2017, EPJ Web of Conf. 145,01001

--- 2019, EPJ Web of Conferences, ? 207, с. 03003. 

Le Cam, L. 1953, On some asymptotic properties of maximum likelihood estimates and related Bayes estimates. Univ. California Publ. Statist, 1, 125-142

Levine, E. S., Blitz, L., \& Heiles, C. 2006a, The Spiral Structure of the Outer Milky Way in Hydrogen, Science, 312, 1773

--- 2006b, The Vertical Structure of the Outer Milky Way H I Disk, ApJ, 643, 881

--- 2006c, The Warp and Spiral Arms of the Milky Way, ArXiv Astrophysics e-prints

Li, W., Yang, H. 2014, Blind source separation in underdetermined model based on local mean decomposition and AMUSE algorithm. Proceedings of the 33rd Chinese Control Conference

Liu, G., Kreinovich, V. 2010, Fast convolution and Fast Fourier Transform under interval and fuzzy uncertainty. Journal of Computer and System Sciences. 76. 73-76

Liu, R.Y. et al., 2016, Phys. Rev. D 94, 043008

Lodders, K. 2003, Solar System Abundances and Condensation Temperatures of the Elements, ApJ, 591, 1220

Lopez-Corredoira, M., Cabrera-Lavers, A., \& Gerhard, O. E. 2005, A boxy bulge in the Milky Way. Inversion of the stellar statistics equation with 2MASS data, Astron. Astrophys., 439, 107

Lorden, G. 1971, Procedures for reacting to a change in distribution. Annals. Math. Statist., 42. 1897-1971

Lovelace R.V.E. 1976, Nature, 262, 649

Malkov, M.A., \& Drury, L.O'C, 2001, Reports on Progress in Physics, 64, 429

--- 2013, Analytic Solution for Self-regulated Collective Escape of Cosmic Rays from Their Acceleration Sites, ApJ, 768, 73

Mashnik, S. G., Sierk, A. J., Van Riper, K. A., et al. 1999, Production and validation of isotope production cross-section libraries for neutrons and protons to 1.7 GeV, in Proc. 4th Workshop on Simulating Accelerator Radiation Environments, Knoxville, Oak Ridge, 1999, ed. T. Gabriel, 151--162, arXiv:nucl-th/9812071

--- 2004, CEM2K and LAQGSM codes as event generators for space-radiation-shielding and cosmic-ray-propagation applications, Adv. Space Res., 34, 1288

--- 2006, CEM03 and LAQGSM03 -- new modeling tools for nuclear applications, Journal of Physics Conference Series, 41, 340

--- 2008, CEM03.03 and LAQGSM03.03 Event Generators for the MCNP6, MCNPX, and MARS15 Transport Codes, ArXiv: 0805.0751

Mathis, J. S., Mezger, P. G., \& Panagia, N. 1983, Interstellar radiation field and dust temperatures in the diffuse interstellar matter and in giant molecular clouds, Astron. Astrophys., 128, 212

Maurin, D., Donato, F., Taillet, R., et al. 2001, Cosmic Rays below Z=30 in a Diffusion Model: New Constraints on Propagation Parameters, ApJ, 555, 585

Men, H., Ferri`ere, K., \& Han, J. L. 2008, Observational constraints on models for the interstellar magnetic field in the Galactic disk, Astron. Astrophys., 486, 819

Merry, R.J.E. 2005, Wavelet Theory and Applications: A literature study. Eindhoven University of Technology

Meyer, J. P., Drury, L. O., \& Ellison, D. C. 1997, Galactic Cosmic Rays from Supernova Remnants. I. A Cosmic-Ray Composition Controlled by Volatility and Mass-to-Charge Ratio, ApJ, 487, 182

Miettinen, J., Illner, K., Nordhausen, K., Oja, H., Taskinen, S. and Theis, F. 2015, Separation of uncorrelated stationary time series using autocovariance matrices, Journal of Time Series Analysis, in print, DOI: 10.1111/jtsa.12159

Mitchell, J., Binns,W. R., Hams, T., et al. 2015, The Heavy Nuclei eXplorer (HNX) Small Explorer Mission, in APS April Meeting 2015, abstract \#S14.004

Moskalenko, I. V., Strong, A. W., \& Reimer, O. 1998a, Diffuse galactic gamma rays, cosmic-ray nucleons and antiprotons, Astron. Astrophys., 338, L75

Moskalenko, I. V., Strong, A. W., 1998b, Production and Propagation of Cosmic-Ray Positrons and Electrons, ApJ, 493, 694

--- 2000, Anisotropic Inverse Compton Scattering in the Galaxy, ApJ, 528, 357

--- 2001, New calculation of radioactive secondaries in cosmic rays, in Proc. 27th Int. Cosmic Ray Conf. (Hamburg), Vol. 5, 1836--1839, arXiv:astro-ph/0106502

Moskalenko, I. V., et al., 2002, Secondary Antiprotons and Propagation of Cosmic Rays in the Galaxy and Heliosphere, ApJ, 565, 280

--- 2003a, Evaluation of Production Cross Sections of Li, Be, B in CR, in Proc. 28th Int. Cosmic Ray Conf. (Tsukuba), Vol. 4, 1969, arXiv:astro-ph/0306367

--- 2003b, Challenging Cosmic-Ray Propagation with Antiprotons: Evidence for a ?Fresh? Nuclei Component?, ApJ, 586, 1050

--- 2004, Diffuse gamma rays, in Astrophysics and Space Science Library, Vol. 304, Cosmic $\gamma$-Ray Sources, ed. K. S. Cheng \& G. E. Romero, 279, arXiv:astro-ph/0402243

--- 2005, Diffuse-ray emission: lessons and perspectives, in American Institute of Physics Conference Series, Vol. 801, Astrophysical Sources of High Energy Particles and Radiation, ed. T. Bulik, B. Rudak, \& G. Madejski, 57--62, arXiv:astro-ph/0509414

--- 2006, Attenuation of Very High Energy Gamma Rays by the Milky Way Interstellar Radiation Field, ApJ, 640, L155

--- 2011, Isotopic Production Cross Sections for CR Applications (ISOPROCS Project), in Proc. 32nd Int. Cosmic Ray Conf. (Beijing), Vol. 6, 277

--- 2013, Isotopic Production Cross Sections (ISOPROCS Project), Proc. 33rd ICRC (Rio de Janeiro), 0823

Muqumov, A.R. \& Galkin V.I. 2018, Estimation of Extensive Air Shower Primary Primary Particle Parameters Using the Data of Particle Detectors of High Mountain Setups, Memoirs of the Faculty of Physics, N 3.

Murphy, E. J., Porter, T. A., Moskalenko, I. V., et al. 2012, Characterizing Cosmic-Ray Propagation in Massive Star-forming Regions: The Case of 30 Doradus and the Large Magellanic Cloud, ApJ, 750, 126

Murphy, R. P., et al. 2016, Galactic Cosmic Ray Origins and OB Associations: Evidence from Super-TIGER Observations of Elements 26Fe through 40Zr, ApJ, 831, 148

Navarro, J. F., Frenk, C. S., White, S. D. M., 1997, A Universal Density Profile from Hierarchical Clustering, ApJ, 490, 493

Niebur, S. M., Scott, L. M., Wiedenbeck, M. E., et al. 2003, Cosmic ray energy loss in the heliosphere: Direct evidence from electron-capture-decay secondary isotopes, Journal of Geophysical Research (Space Physics), 108, 8033

Nielsen, M.A. 2015, Neural Networks and Deep Learning. Determination Press

Nolan, P. L., Abdo, A. A., Ackermann, M., et al. 2012, \fermi{} Large Area Telescope Second Source Catalog, ApJS, 199, 31

Orlando, E., \& Strong, A. 2013, Galactic synchrotron emission with cosmic ray propagation models, MNRAS, 436, 2127

Ostapchenko, S. 2011, Monte Carlo treatment of hadronic interactions in enhanced Pomeron scheme: QGSJET-II model, Physical Review D 83, 014018

Ostapchenko, S., Bleicher, M. 2019, Taming the Energy Rise of the Total Proton-Proton Cross-Section, Universe, 5, 106

Page, E.S. 1954, Continuous inspection schemes. Biometrica. 42(1). 100-115

Pajunen, P. 1998, Blind source separation using algorithmic information theory. Neurocomputing, 22:35-48

Panov, A. D., et al., 2007, Elemental energy spectra of cosmic rays from the data of the ATIC-2 experiment, Bulletin of the Russian Academy of Sciences: Physics, 71, 494

--- 2009, Energy spectra of abundant nuclei of primary cosmic rays from the data of ATIC-2 experiment: Final results, Bulletin of the Russian Academy of Sciences, Physics, 73, 564

Parker, E. N. 1965, The passage of energetic charged particles through interplanetary space, Planet. Space Sci., 13, 9

Parzen E. 1962, On estimation of a probability density function and mode. Ann. Math. Statist. 33:3.1065-1076

Pchelintsev, E.A. 2011, The James-Stein procedure for a conditional Gaussian regression (in Russian). Tomsk State University Journal of Mathematics and Mechanics. 4(16). 6--17

Picozza, P., Galper, A. M., Castellini, G., et al. 2007, PAMELA A payload for antimatter matter exploration and light-nuclei astrophysics, Astroparticle Physics, 27, 296

Planck Collaboration, Adam, R., Ade, P. A. R., et al. 2015a, Planck 2015 results. X. Diffuse component separation: Foreground maps, ArXiv:1502.01588

--- 2015b, Planck 2015 results. XXV. Diffuse low-frequency Galactic foregrounds, ArXiv: 1506.06660

--- 2016, Planck intermediate results. XLII. Large-scale Galactic magnetic fields, ArXiv: 1601.00546

Polikar, R. 1999, The wavelet tutorial. URL: http://users.rowan.edu/ polikar/WAVELETS/WTtutorial.html

Pollak, M. 1985a, Optimal detection of a change in distribution. Ann. Statist. 13. 206-227

--- 1985b. A diffusion process and its application to detecting a change in the drift of Brownian motion process. Biometrika. 72. 267-280

Porter, T. A., \& Strong, A. W. 2005, A new estimate of the Galactic interstellar radiation field between 0.1um and 1000um, Proc. 29th Int. Cosmic Ray Conf. (Pune), 4, 77

--- 2006, Inverse Compton Emission from Galactic Supernova Remnants: Effect of the Interstellar Radiation Field, ApJ, 648, L29

--- 2008, Inverse Compton Origin of the Hard X-Ray and Soft $\gamma$-Ray Emission from the Galactic Ridge, ApJ, 682, 400

--- 2015, The FRaNKIE Code: a Tool for Calculating Multi-Wavelength Interstellar Emissions in Galaxies, Proc. 34th ICRC (Hague), 908

--- Porter, T. A., Johanneson, G., Moskalenko, I. V., 2017, High-energy Gamma Rays from the Milky Way: Three-dimensional Spatial Models for the Cosmic-Ray and Radiation Field Densities in the Interstellar Medium, ApJ 846, 67

Postnikov E.B., Grinyuk A.A., Kuzmichev L.A., Sveshnikova L.G. 2017, Journal of Physics: Conference Series, v 798, ? 1, с. 012030

Prantzos, N., et al., 2011, The 511 keV emission from positron annihilation in the Galaxy, Reviews of Modern Physics, 83, 1001

Press, W. H., Teukolsky, S. A., Vetterling, W. T., et al. 1992, Numerical recipes in FORTRAN. The art of scientific computing (Cambridge University Press, 2nd ed.)

Principe, G., Malyshev, D., Ballet, J., and Funk, S., 2018, The first catalog of \fermilat{} sources below 100 MeV, Astronomy and Astrophysics, 618, A22

Protheroe, R. J. 1982, On the nature of the cosmic ray positron spectrum, ApJ, 254, 391 

Prosin V.V., et al., (Tunka Collaboration), 2014, NIM A, 756, 94

Pshirkov M. S., Tinyakov P. G., Kronberg P. P., et al. 2011, Deriving the Global Structure of the Galactic Magnetic Field from Faraday Rotation Measures of Extragalactic Sources. The Astrophysical Journal. 738. 192

--- 2016, New Limits on Extragalactic Magnetic Fields from Rotation Measures. Physical Review Letters. 116. 191302

Ptitsyna K.B., Troitskiy S.V., 2010, Uspekhi Phys. Nauk, 53, 691. (in Russian)

Ptuskin, V. S., \& Soutoul, A. 1998, Decaying cosmic ray nuclei in the local interstellar medium, Astron. Astrophys., 337, 859

--- 2001, Propagation, Confinement Models, and Large-Scale Dynamical Effects of Galactic Cosmic Rays, Space Sci. Rev., 99, 281

--- 2006, Dissipation of Magnetohydrodynamic Waves on Energetic Particles: Impact on Interstellar Turbulence and Cosmic-Ray Transport, ApJ, 642, 902

--- 2013a. Adv. Space Res 51, 315

--- 2013b. ApJ, 763, 47

V. Ptuskin V., V. Zirakashvili V., et al. 2010, Astrophysical Journal, 718:31--36.

Puigt, M., Berne, O., Guidara, R., Deville, Y., Hosseini, S., Joblin, C. 2009, Cross-validation of blindly separated interstellar dust spectra. Proc. of ECMS 2009, 41-48, Mondragon, Spain, July 8-10, 2009

Pyzdek, Th. 1991, Johnson Control Charts. Quality, February. 41

Rauch, B. F., Link, J. T., Lodders, K., et al. 2009, Cosmic Ray origin in OB Associations and Preferential Acceleration of Refractory Elements: Evidence from Abundances of Elements 26Fe through 34Se, ApJ, 697, 2083

Rawlins, K., et al., 2016, Cosmic ray spectrum and composition from three years of IceTop and IceCube, Journal of Physics: Conference Series, 718, 052033

Riedmiller, M. 1994, Advanced supervised learning in multi-layer perceptrons. From backpropagation to adaptive learning algorithms. Computer Standards and Interfaces 16(5), 265- 278

Robitaille, T. P., Churchwell, E., Benjamin, R. A., et al. 2012, A self-consistent model of Galactic stellar and dust infrared emission and the abundance of polycyclic aromatic hydrocarbons, Astron. Astrophys., 545, A39

Robbins, H. 1956, An Empirical Bayes Approach to Statistics. Berkeley Symp. on Math. Statist. and Prob. Proc. Third Berkeley Symp. on Math. Statist. and Prob., Vol. 1 (Univ. of Calif. Press, 1956), 157-163

Rosenblatt M. 1956, Remarks on some nonparametric estimates of a density functions. Ann. Math. Statist. 27:3. 832-837

Rubtsov G.I., Troitsky S.V. 2014, Breaks in $\gamma$-ray spectra of distant blazars and transparency of the Universe. JETP Lett. 100. 355

Ruiz de Austri, R., Trotta, R., \& Roszkowski, L. 2006, A Markov chain Monte Carlo analysis of the CMSSM, Journal of High Energy Physics, 5, 2

Saftly, W., Baes, M., \& Camps, P. 2014, Hierarchical octree and k-d tree grids for 3D radiative transfer simulations, Astron. Astrophys., 561, A77

Schure K.M., Bell A.R.,O'Cdrury L., Bykov A.M. 2012, Space Science Reviews, 173, 491 Seo, E. S., \& Ptuskin, V. S. 1994, Stochastic reacceleration of cosmic rays in the interstellar medium, ApJ, 431, 705

---  2014, Cosmic Ray Energetics And Mass for the International Space Station (ISS- CREAM), Advances in Space Research, 53, 1451

Serpico, P.D., 2018, Entering the cosmic ray precision era, J. Astrophys. Astr. 39, 41

Sfeir, D. M., Lallement, R., Crifo, F., et al. 1999, Mapping the contours of the Local bubble: preliminary results, Astron. Astrophys., 346, 785

Sheluhin, O. I. Filinova A.S. 2007, Detection of network anomaly bursts of traffic by the method of the disorder of Brodsky-Darkhovsky (in Russian). T-Comm: telecommunication and transport. 7(10). 116--119

Shiryaev, A.N. 1963, On optimum methods in quickest detection problems (in Russian). Theor. Probab. Appl. 8. 22-46

--- 1965, Some explicit formulae in a problem on ``disorder'' (in Russian). Theory Probab. Appl. 10:2. 348-354

Silberberg, R., Tsao, C. H., \& Barghouty, A. F. 1998, Updated Partial Cross Sections of Proton-Nucleus Reactions, ApJ, 501, 911

Silk, J., 2018, Molecular Ionization Rates and Ultracompact Dark Matter Minihalos, Phys. Rev. Lett., 121, 231105

Skilling, J. 2004, Nested Sampling, in American Institute of Physics Conference Series, Vol. 735, American Institute of Physics Conference Series, ed. R. Fischer, R. Preuss, \& U. V. Toussaint, 395--405

--- 2006, Nested sampling for general Bayesian computation, Bayesian Analysis, 1, 833 Sodroski, T. J., Odegard, N., Arendt, R. G., et al. 1997, A Three-dimensional Decomposition of the Infrared Emission from Dust in the Milky Way, ApJ, 480, 173

Soutoul, A., Legrain, R., Lukasiak, A., et al. 1998, Evidence from Voyager and ISEE-3 spacecraft. Data for the decay of secondary K-electron capture isotopes during the propagation of cosmic rays in the Galaxy, Astron. Astrophys., 336, L61

Stecker, F. W. 1971, Cosmic gamma rays (Baltimore: Mono Book Co.)

Stein, C. 1956, Inadmissibility of the usual estimator for the mean of a multivariate distribution. Proc. Third Berkeley Symp. Math. Statist. Prob. 1, 197---206

--- 1981. Estimation of the mean of a multivariate normal distribution. Ann. Statist. 9(6) 1135-1151

Stone, E. C., Vogt, R. E., McDonald, F. B., et al. 1977, Cosmic ray investigation for the Voyager missions: Energetic particle studies in the outer heliosphere -- and beyond, Space Sci. Rev., 21, 355

--- 1998, The Cosmic-Ray Isotope Spectrometer for the Advanced Composition Explorer, Space Sci. Rev., 86, 285

--- 2013, Voyager 1 Observes Low-Energy Galactic Cosmic Rays in a Region Depleted of Heliospheric Ions, Science, 341, 150

Strang, G., Nguyen, T. 1997. Wavelets and Filter Banks. Wellesley-Cambridge Press, second edition.

Strong, A. W., \& Mattox, J. R. 1996, Gradient model analysis of EGRET diffuse Galactic -- ray emission., Astron. Astrophys., 308, L21

Strong, A. W., Moskalenko, I. V., 1998, Propagation of Cosmic-Ray Nucleons in the Galaxy, ApJ, 509, 212

--- 2001, Models for galactic cosmic-ray propagation, Adv. Space Res., 27, 717

Strong, A. W., Moskalenko, I. V., Reimer, O. 2000, Diffuse Continuum Gamma Rays from the Galaxy, ApJ, 537, 763

--- 2004, Diffuse Galactic Continuum Gamma Rays: A Model Compatible with EGRET Data and Cosmic-Ray Measurements, ApJ, 613, 962

Strong, A. W., Moskalenko, I. V., Ptuskin, V. S. 2007, Cosmic-Ray Propagation and Interactions in the Galaxy, Ann. Rev. Nucl. Part. Sci., 57, 285

Strong, A. W., et al., 2010, Global Cosmic-ray-related Luminosity and Energy Budget of the Milky Way, ApJ, 722, L58

--- 2011, The interstellar cosmic-ray electron spectrum from synchrotron radiation and direct measurements, Astron. Astrophys., 534, A54

Su, M., Slatyer, T. R., \& Finkbeiner, D. P. 2010, Giant $\gamma$-ray Bubbles from \fermilat{}: Active Galactic Nucleus Activity or Bipolar Galactic Wind? ApJ, 724, 1044

Sun, X. H., Reich, W., Waelkens, A., et al. 2008, Radio observational constraints on Galactic 3D emission models, Astron. Astrophys., 477, 573

Svesnikova L.G et al, 2019: Л.Г. Свешникова и др.(TAIGA collaboration). Первый сезон работы гибридной черенковской установки  TAIGA. Известия РАН, сер. физическая, N8. 

Tauber, J. A., Mandolesi, N., Puget, J.-L., et al. 2010, Planck pre-launch status: The Planck mission, Astron. Astrophys., 520, A1

Taylor A.M. et al., 2011, Extragalactic magnetic fields constraints from simultaneous GeV- TeV observations of blazars. A\&A. 529. A144

Theis, F., Meyer-Bese, A., Lang, E. 2004, Second-Order Blind Source Separation Based on Multi-dimensional Autocovariances. Independent Component Analysis and Blind Signal Separation. 726?733

Thoudam, S. et al. 2015, astro--ph/1506.09134

--- 2016, Cosmic-ray energy spectrum and composition up to the ankle: the case for a second Galactic component. A\&A.595. id.A33

Titarenko, Y. E., Batyaev, V. F., Titarenko, A. Y., et al. 2008, Cross sections for nuclide production in a Fe56 target irradiated by 300, 500, 750, 1000, 1500, and 2600 MeV protons compared with data on a hydrogen target irradiated by 300, 500, 750, 1000, and 1500 MeV/nucleon Fe56 ions, Phys. Rev. C, 78, 034615

--- 2011a, Measurement and simulation of the cross sections for nuclide production in 56Fe and natCr targets irradiated with 0.04- to 2.6-GeV protons, Physics of Atomic Nuclei, 74, 523

--- 2011b, Verification of high-energy transport codes on the basis of activation data, Phys. Rev. C, 84, 064612

Tluczykont M. et al., AstropaticlePhys., 2014, 56,42

Tokuno H. et al. (The Telescope Array Collaboration), 2008, APh 29, 453

--- 2012, NIM A 676, 54

Trotta, R. 2008, Bayes in the sky: Bayesian inference and model selection in cosmology, Contemporary Physics, 49, 71

--- 2011, Constraints on Cosmic-ray Propagation Models from A Global Bayesian Analysis, ApJ, 729, 106

Vautard, R., Ghil, M. 1989. Singular spectrum analysis in nonlinear dynamics, with applications to paleoclimatic time series. Physica D. 35. 395-424

Vallee J. P. 2004, Cosmic magnetic fields -- as observed in the Universe, in galactic dynamos, and in the Milky Way. New Astronomy Reviews. 48. 763--841

--- 2014, The Spiral Arms of the MilkyWay: The Relative Location of Each Different Arm Tracer within a Typical Spiral Arm Width, AJ, 148, 5

Vladimirov, A. E., Digel, S. W., Johannesson, G., et al. 2011, GALPROP WebRun: An internetbased service for calculating galactic cosmic ray propagation and associated photon emissions, Computer Physics Communications, 182, 1156

--- 2012, Testing the Origin of High-energy Cosmic Rays, ApJ, 752, 68

Vorobeychikov, S. E. 1988, On the detection of a change in the mean of a sequence of random variables. Automation and Remote Control. 59:3. 50-56

--- 2002, Detection of the change point in a sequence of independent random variables (in Russian). Journal of Communications Technology and Electronics. 47:10. 1198-- 1203

--- 2014, An Algorithm for the Automatic Detection of Inclusions in an Inspected Object with a Scanning Digital X-Ray Imaging System (One-Dimensional Variant). Russian Journal of Nondestructive Testing. 50(6). 359

--- 2015, A Study of Two Image-Recognition Algorithms for the Classification of Flaws in a Test Object According to Its Digital Image. Russian Journal of Nondestructive Testing. 51(10). 644

Vovk Te. et al. 2012, \fermi{}/LAT Observations of 1ES 0229+200: Implications for Extragalactic Magnetic Fields and Background Light. ApJLet. 747. L14

Wainscoat, R. J., Cohen, M., Volk, K., et al. 1992, A model of the 8-25 micron point source infrared sky, ApJS, 83, 111

Waxman E. 2004, New J.Phys. 6, 140

Webber, W. R., Soutoul, A. 1998, A Study of the Surviving Fraction of the Cosmic-Ray Radioactive Decay Isotopes 10Be, 26Al, 36Cl and 54Mn as a Function of Energy Using the Charge Ratios Be/B, Al/Mg, Cl/Ar, and Mn/Fe Measured on HEAO3, ApJ, 506, 335

--- 2003, Updated Formula for Calculating Partial Cross Sections for Nuclear Reactions of Nuclei with $Z < 29$ and $E > 150$ MeV/Nucleon in Hydrogen Targets, ApJS, 144, 153

Wellisch, H. P., Axen, D. 1996, Total reaction cross section calculations in proton-nucleus scattering, Phys. Rev. C, 54, 1329

Wibig, T., Wolfendale, A. W. 2005, At what particle energy do extragalactic cosmic rays start to predominate? J. Phys. G: Nucl. Part. Phys. 31. 255

--- 2007, Ultra High Energy Cosmic Rays. J. Phys. G: Nucl. Part. Phys. 34.1891

Winkler, C., Courvoisier, T. J.-L., Di Cocco, G., et al. 2003, The INTEGRAL mission, Astron. Astrophys., 411, L1

Winkler, C., Diehl, R., Ubertini, P., Wilms, J. 2011, INTEGRAL: Science Highlights and Future Prospects, Space Science Reviews, 161, 149

Yao J. M., Manchester R. N., Wang N. 2017, A New Electron-density Model for Estimation of Pulsar and FRB Distances. The Astrophysical Journal. 835. 29

Yoon, Y. S., Ahn, H. S., Allison, P. S., et al. 2011, Cosmic-ray Proton and Helium Spectra from the First CREAM Flight, ApJ, 728, 122.

--- 2017, Proton and Helium Spectra from the CREAM-III Flight, ApJ 839, 5

Zatsepin, G. T. and Kuzmin V., 1966,Upper Limit of the Spectrum of Cosmic Rays. Journal of Experimental and Theoretical Physics Letters 4: 78.

Zatsepin V.I. et al. 1990, Proc. 21st ICRC, 3, 81

--- 1994, Yadern. Phys., 57, 684. (in Russian)

Zeitlin, C., Miller, J., Guetersloh, S., et al. 2011, Fragmentation of N14, O16, Ne20, and Mg24 nuclei at 290 to 1000 MeV/nucleon, Phys. Rev. C, 83, 034909

Ziehe, A., Kawanabe, M., Harmeling, S., Muller, K-R. 2003, Blind separation of post- nonlinear mixtures using linearizing transformations and temporal decorrelation. Journal of Machine Learning Research. 4. 1319-1338

Zirakashvili, V.N., Ptuskin, V.S., 2008, Diffusive Shock Acceleration with Magnetic Amplification by Nonresonant Streaming Instability in Supernova Remnants, ApJ 678, 939

--- 2012, Astroparticle Physics, 39, 12

--- 2014, ApJ, 785, 130

--- 2017, arXiv:1701.00844

--- 2918a, Cosmic Rays and Nonthermal Radiation in Middle-Aged Supernova Remnants?, Astronomy Letters, 44, 769 

--- 2018b, Cosmic ray acceleration in magnetic circumstellar bubbles, Astroparticle Physics, 98, 21

Zou, J. 2007, A sublinear algorithm for the recovery of signals with sparse Fourier transform when many samples are missing. Appl. Comput. Harmon. Anal. 22. 61--77

